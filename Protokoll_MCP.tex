\documentclass[a4paper, 10pt]{scrartcl}

\usepackage{a4wide}
\usepackage{float}

\usepackage[ngerman]{babel}
\usepackage[utf8]{inputenc}
\usepackage[pdftex]{graphicx} %%Grafiken in pdfLaTeX
\usepackage{amsmath}
%\usepackage{amssymb}
%\usepackage{empheq}
\usepackage{mathrsfs}
\usepackage{amsthm}
\usepackage{stmaryrd }   %Symbole, z.B. Blitz
\usepackage{booktabs}
\usepackage{multirow}		%Zum Zusammenfassen mehrere Zeilen einer Tabelle
%\usepackage{amsfonts}
\usepackage{hyperref}

%\usepackage{pictex}




% Zusatzpakete verbatim und moreverb: listing-Umgebung
\usepackage{verbatim, moreverb}
\usepackage{color}
\definecolor{darkred}{rgb}{0.5,0,0}
\usepackage{listings}		%Für Quellcode

\lstloadlanguages{Java, XML}
\lstset{language=Java}
\lstset{
basicstyle=\scriptsize\ttfamily, % alle listings scriptsize drucken (kann man gerade noch lesen) und Schreibmaschinenschrift für alles
keywordstyle=\color{darkred}\bfseries, % Schlüsselwörter fett und dunkelrot drucken
commentstyle=\color{blue}, % Kommentare blau drucken
showstringspaces=false, % Strings im Code ohne Kenntlichmachung von Leerzeichen
breaklines=true,
frame=lines,
numbers=left,
numberstyle=\tiny
}




%  \usepackage{SIunits}
%  \usepackage[version=3]{mhchem}




%%%%%%%%%%%%%%%%%%%%%%%%%%%%%%%%%

% Stichwortverzeichnis 
\usepackage{makeidx}
\makeindex
%
\newcommand{\idx}[1]{\index{#1} #1}
%indizierte Quelltextline - kleiner Fehler in der Darstellung im Index: Schrift lässt sich nicht anpassen, da Eintrag sonst nicht erscheint.
\newcommand{\lstx}[1]{\index{\lstinline$#1$} \lstinline[basicstyle=\normalsize\ttfamily]$#1$}


%Absatzeinstellungen
\setlength{\parindent}{0cm}
\setlength{\parskip}{0.3cm}

%Gleitobjekteinstellungen
\setcounter{bottomnumber}{2}
\renewcommand{\textfraction}{0}
\renewcommand{\bottomfraction}{0.7}

%Abstand zw. Bild und Bildunterschrift
\setlength{\abovedisplayshortskip}{0pt}



\title{Praktikum Mobile Computing}
\author{RK}		%Fügt eure Namen am Ende hinzu


\begin{document}

\begin{titlepage}
   
   \begin{center}
   \vspace*{4cm}
      \huge{Praktikum Mobile Computing}\\
      \vspace{1cm}
      
      \Large{18.09. – 29.09.2017}
      
      \vfill
      \Large{RK, }\\		%Namen hinzufügen
      \large{@}\\	%Mailadressen hinzufügen
      \vspace{1cm}
      \today
      
      \vspace*{2cm}
      
      
   \end{center}
   
\end{titlepage}

\tableofcontents

\newpage

\begin{abstract}
Mapbox GPS-App
\end{abstract}

\newpage

\section{Einleitung}

Mindest API: Android 5.0 

Problem: Couchbase will mindestens API 25 / 7.0 Nougat, das haben wir aber gar nicht.
Anfrage an Betreuer

\section{Datenbank}

Couchbase:  Couchbase Lite, embedded NoSQL database

Speichert Datensätze im JSON-Format, KEy-Value-Paare

``type'': ``bezeichner'' um den Typ festzulegen

Erstellungszeit: ``createdAt'':  ``YYYY-MM-DDThh:mm:ssZ'' (ISO-8601 format)

DocumentID: \lstinline{"_id"} : `` '' Üblich ist username.XXX wegen Access Controll

\section{Sammlung ToDo}

Startbildschirm, von da aus zu versch. Activities:
\begin{itemize}
\item Weg aufzeichnen
\item Karte anzeigen mit Pos.
\item Datenbank anschauen + Route verwalten
\item Route auf der Karte anzeigen
\end{itemize}

\begin{itemize}
 \item Icon erstellen: Karte mit Route
\item Name?
\item Standpunkt per GPS erfassen
\item Zusatzgeräte - Datenformat?
\item Häufigkeit/Genauigkeit GPS?
\item Einzeldaten - wie viele?
\item Umwandeln in Polygone um Datenmenge zu reduzieren
\item Wege Erfassen
\item Datenbank: CouchBaseLight(?)
\end{itemize}





\section{Quelltextbeispiele}

\subsection{Unterüberschrift}

\subsection{Beispiel Quelltext}

\begin{lstlisting}
package com.example.android.helplearntolearn;

import android.annotation.SuppressLint;
import android.support.v7.app.ActionBar;
import android.support.v7.app.AppCompatActivity;
import android.os.Bundle;
import android.os.Handler;
import android.view.MotionEvent;
import android.view.View;

/**
 * An example full-screen activity that shows and hides the system UI (i.e.
 * status bar and navigation/system bar) with user interaction.
 */
public class FullscreenActivity extends AppCompatActivity {
    /**
     * Whether or not the system UI should be auto-hidden after
     * {@link #AUTO_HIDE_DELAY_MILLIS} milliseconds.
     */
    private static final boolean AUTO_HIDE = true;

    /**
     * If {@link #AUTO_HIDE} is set, the number of milliseconds to wait after
     * user interaction before hiding the system UI.
     */
    private static final int AUTO_HIDE_DELAY_MILLIS = 3000;
\end{lstlisting}








\end{document}
